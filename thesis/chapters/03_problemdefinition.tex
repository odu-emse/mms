\chapter{Problem Definition}
\label{ch:problemdefinition}

% a.	“How can we as educators, advocates, and engineers evolve education?”
%   i.	By delivering content related to the student's interests and being aware of their personal experiences (professionally or otherwise) as domain experience for a given topic.
%   ii.	By teaching in a method that facilitates the most efficient learning for the individual student. 
%     1.	This causes its own pitfall on a large scale since instructors will need to either teach in multiple styles or contract co-instructors.
% b.	“How can this project facilitate educational evolution?”
%   i.	It forces instructors to modularize their courses into small chunks of learning outcomes or topics.
%     1.	This will cut down on repetition that is introduced by courses that cover topics that other courses have taught or touched on.
%     2.	For example: “We learned about the Python programming language in Data Visualizations, but by that time my previously completed course already taught me.”	
%   ii.	Match the student's learning objectives to topics that most closely correlate with their degree wants.


To grasp the duality of the problem, we need to ask ourselves two questions from different perspectives.

\paragraph{How can we evolve education as educators, advocates, and engineers?} 

We must start by delivering content related to the student's interests closely and by considering their personal experiences (professionally or otherwise) as domain experience for a given topic. Furthermore, we must teach in a method that facilitates the most efficient learning for the individual student. This causes its own pitfall on a large scale since instructors will need to either teach in multiple styles or contract co-instructors.


\paragraph{How can this project facilitate educational evolution?}

This project forces instructors to modularize their courses into small chunks of learning outcomes rather than structuring their courses in an old pyramid format. From there, the system should facilitate the reduction of repetition that is introduced by courses that require the same baseline knowledge or have the same fundamental concepts. For example:" We learned about the ins and outs of the Python programming language in Data Visualizations, but by that time, my previously completed course already taught me." Once Modules are tagged with their respective learning outcomes, the system will be able to optimize their learning paths based on pre-existing knowledge of the topic at hand. This optimization is made possible by matching the student's learning objectives to topics most closely correlated with their degree wants.