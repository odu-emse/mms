\chapter{TF-IDF Architecture}
\label{ch:tfidfarchitecture}

One of the methods implemented uses TF-IDF values to convert the modules to text vectors and then K-Means to predict the best cluster to associate the module with. These clusters will be referred to as Collections. After this, we used PCA to plot the text vectors on a 2D graph to represent how modules differ visually. The key takeaway from this method was the need to have a custom "stop word" bank since there were a lot of repetitive words throughout all modules that didn't necessarily describe the given module. To weed out these words, we opted to generate word clouds of each cluster to know what words the key phrases for each cluster are. Word stemming turned out to be trickier than expected, as sometimes stemming words made the sentence lose its context and negatively influenced the final outcome of the clusters. This was improved upon with the new architecture that will be discussed later. When it came to making the algorithm more context-aware, we opted to use a Part-Of-Speech tagging library. In corpus linguistics, part-of-speech tagging, also called grammatical tagging, is the process of marking up a word in a text as corresponding to a particular part of speech based on its definition and context. When combining the stop word removal, word stemming, and POS tagging, the final text content was cut down in context and natural patterns as well.
