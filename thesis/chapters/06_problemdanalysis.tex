\chapter{Problem Analysis}
\label{ch:problemanalysis}

% a.	To understand the problem we are trying to solve here, we have to go to the ground floor of the pyramid, the content. We have to cut down on the repetitive topics being covered in different classes and put the time saved to better use. Maybe students don't have an interest in particular topics of a course. We should ask ourselves, is this topic necessary or trivial to the learning outcome of the course? If it isn't necessary, then why not offer topics that might interest the learner while still conveying the same concepts? This isn't supposed to insult or undermine instructors but rather help them understand how to structure their courses in a manner that is modularizable.
% b.	Obviously, in order to facilitate anything remotely similar, institutions would have to turn a chunk of their funds and efforts towards rebuilding and re-recording legacy courses. This would, of course, need to be supported by instructional designers with expertise in creating online learning materials and modular courses. 
% c.	Unless this is met, the evolution of education will be stagnant. This is backed up by my experience with having an extremely limited sample size of only 157 videos, out of which only 63 came from a genuinely modularized course. 
  % i.	A truly modularized course would consist of video materials that are between 5 - 15 minutes long and deliver the same learning outcomes as a whole as the course.
  % ii.	The more segmented these courses become, the more accurate the system would be, as it would have a more significant number of linkages that could be made for solid clustering ability and a more extensive sample of documents to embed and gain context. 
  % iii.	It's been evident throughout this project that a transcript of one large video and five transcripts of 5 videos will not produce the same results. Even with GTP-3.5, the context awareness of the model drops significantly the longer a text document is. 


To understand the problem we are trying to solve here, we have to go to the ground floor of the pyramid, the content. We have to cut down on the repetitive topics being covered in different classes and put the time saved to use better. Students may not have an interest in particular topics of a course. We should ask ourselves, is this topic necessary or trivial to the learning outcome of the course? If it is not necessary, then why not offer topics that might interest the learner while still covering the same concepts in the course as a whole? This is not supposed to undermine or insult instructors but rather help them understand how to structure their courses in a manner that is modularizable. In order to facilitate anything remotely similar, institutions would have to turn a chunk of their funds and efforts towards rebuilding and re-recording legacy courses. This needs to be supported by instructional designers with expertise in creating online learning materials and modular course structures. Unless this is met, the evolution of data-driven online education will be stagnant. This is backed up by my experience with an extremely limited sample size of 157 videos, of which only 63 came from a genuinely modularized course. A truly modularized course would consist of video materials that are between 5 - 15 minutes long and deliver the same learning outcomes as a whole as the course. The more segmented these courses become, the more accurate the system would be, as it would have a more significant number of linkages that could be made for solid clustering ability and a more extensive sample of documents to embed and gain context of. It has been evident throughout this project that a transcript of one large video and five transcripts of 5 videos will produce different results. Even with GTP-3.5, the context awareness of the model drops significantly the longer a text document is.