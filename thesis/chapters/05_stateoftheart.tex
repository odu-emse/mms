\chapter{State of the Art}
\label{ch:stateoftheart}

% a.	“Advancements in artificial intelligence (AI) and adaptive technologies enable educators to meet learners where they are, with experiences tailored to their needs.”
% i.	This quote by Google's research paper puts the need for AI in education in perspective. If arguably the most prominent player in tech recognizes and funds the expansion of online learning, why can't institutions where all the technological advancements start adapting to the changing climate?
% b.	“If we had been spending time knowing our children and our staff and designing schools for them, we might not be feeling the pain in the way we are. I think we're learning something about what the real work of school is about.”
% i.	The following quote by the Harvard Graduate School of Education recognizes the need to adapt schools to the student and stop treating school as a one size fits all concept. By leveraging modern technologies, we can draw profiles of each student and find out how to best match their needs for learning the appropriate information. 
% c.	Even the UN agrees that accessible education needs to be at the forefront of every philanthropist's mind. COVID-19 didn't create the need for adaptable online learning. The society we live in that's “never sleeping” and constantly evolving is what pushed the need for adaptable and accessible learning.

Education has been making big waves in private and public research, especially in the past few years. The COVID-19 pandemic has forced schools to adapt to the new climate and start teaching online. This has been a big challenge for many schools, as they were not prepared for such a drastic change. Nevertheless, education was not flourishing even before the pandemic. This event was a wake-up call for the education system to adapt to the changing climate.

\textbf{"Advancements in artificial intelligence (AI) and adaptive technologies enable educators to meet learners where they are, with experiences tailored to their needs."}\cite{google:aiineducation}

The following quote from Google's research paper puts the need for AI in education in perspective. If arguably the most prominent player in tech recognizes and funds the expansion of online learning, why cannot institutions where all the technological advancements stem from start adapting to the changing climate?

\textbf{"If we had been spending time knowing our children and our staff and designing schools for them, we might not be feeling the pain in the way we are. I think we're learning something about what the real work of school is about."}\cite{harvard:education}

The following quote by the Harvard Graduate School of Education recognizes the need to adapt schools to the student and stop treating school as a one size fits all concept. By leveraging modern technologies, we have the ability to draw up profiles of each student and find out how to best match their needs for learning the appropriate information. 

\textbf{"We already faced a learning crisis before the pandemic."}\cite{un:education}

Even the United Nations agrees that accessible education needs to be at the forefront of every philanthropist's mind. COVID-19 did not create the need for adaptable online learning. The society we live in that is "never sleeping" and constantly evolving is what pushed the need for adaptable and accessible online education.