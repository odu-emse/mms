\abstract{
The field of education is constantly evolving, and higher education institutions are always looking for ways to improve the student experience regardless of their respective delivery methods. Educators have been enhancing their delivery methods since the 1980s to align with students' needs and learning styles as closely as possible. Throughout the past two decades, as education shifted its focus to remote delivery methods (satellite programs, CD-ROM deliveries), students lost individuality, and educators became distant from their course content flexibility. The partnership between technology and education has allowed the delivery of course content at a rapid pace but sacrificed content flexibility which the educator previously controlled. The tool that could fix the separation between courses and student learner preferences is used by most humans daily. Such improvement is implementing a machine learning system that helps students select courses that align with their previous professional experiences and learner preferences. Such a system can not only make course selection automated for students, but it can also improve the effectiveness of the education provided by customizing the course contents around the students' needs. 
}
